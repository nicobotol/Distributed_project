\section{Conclusion}
The project objectives have been successfully fulfilled since the parachute storm's centroid has been correctly driven from the initial point to the ground target. The prescribed centroid trajectory has been computed with an optimal controller, while two ways to coordinate the agents' movement have been proposed. The employed parachutes were a simple linear model and a more complicated unicycle-like nonlinear one, so two different control laws have been studied for them. Finally, no collisions between agents have been reported thanks to employing a control law based on the Voronoi tessellation. \\
Some aspects that can be included in the feature are:
\begin{itemize}
    \item The presence of a minimum forward speed at which the parachute has always to fly;
    \item Consider the forces applied by real actuators instead of their overall effects on the velocities
    \item The possibility of unidirectional communications, leading to non-symmetric adjacency matrix; 
    \item Implementation of a fully 3D Voronoi cell able to manage all the cases of inclusion of agents in all directions;
    \item Not discharging the self-localization information reached after the WLS round before using it in the new KF/EKF, this may be done with some more advanced localization algorithm, such as the Cooperative localization as described in \cite{b14}.
\end{itemize}