\begin{abstract}
This project deals with the localization and control of a swarm of parachutes intended to be dropped from a certain height and autonomously reach a desired location on the ground. In particular, each agent controls its movement to drive the centroid of the storm towards the desired final location.\\
The parachutes are firstly modelled as devices able to move independently along the three cartesian axes (thanks to an actuator in each of them), while later, they are modelled as unicycles with forward velocity, rotation and falling velocity controls. \\
The main features that each parachute has to implement are self-localization in the space (via absolute and relative position measurement), the optimal control to design the trajectory to be followed, and collision avoidance.\\
The simulation results show that the parachutes can accomplish their task and reach the ground without any collision, even in the presence of noises in the inputs, dynamics and GPS tracking.
\end{abstract}