\section{Adopted models}
\subsection{Simplified parachute model}
The considered parachutes are made of two main parts: a sail with finite radial dimensions and height and a payload with dimension negligible compared to the first. The latter is placed below the former and carries also the communication, measurement and control systems.\\
The most simple dynamical parachute model is linear and considers three actuators able to move the device in space, while gravity acts as non-controllable input along the vertical direction.
% Matrix expression placed in another file for clarity
\begin{gather}
\label{eq:model_1}
    x_{i+1} = Ax_i + Bu + G\nu \\
    \begin{bmatrix}
    x_{i+1}\\
    y_{i+1}\\
    z_{i+1}
    \end{bmatrix}=
    %\begin{bmatrix}
    I_3
    %\end{bmatrix}
    \begin{bmatrix}
    x_{i}\\
    y_{i}\\
    z_{i}
    \end{bmatrix} +
    \Delta t
    %\begin{bmatrix}
    I_3
    %\end{bmatrix}
    \begin{bmatrix}
    v_x\\
    v_y\\
    v_z
    \end{bmatrix} + %\notag \\
    \begin{bmatrix}
    1&0&0&0\\
    0&1&0&0\\
    0&0&1&\Delta t
    \end{bmatrix}
    \begin{bmatrix}
    \nu_x\\
    \nu_y\\
    \nu_z\\
    \bar{v_z}
    \end{bmatrix}
     \notag
\end{gather}
The uncertainties $\nu_x$, $\nu_y$, and $\nu_z$ model the wind disturbances and the error along the three directions ($\left[\nu_x \, \nu_y \,\nu_z \right]\sim\mathcal{N}\left(0_3, L\right)$), while $\bar{v_z}$ is the velocity at which the parachutes fall due to the gravity action. 

$\bar{v_z}$  is the non-controllable input related to the gravity that pulls the parachute to the ground.\\
In this way, the total velocity in the z direction has been split into a non-controllable part, which is the one related to gravity, and into a controllable one. A parachute can reduce its falling velocity by breaking until a minimum one (necessary to avoid the closure of the sail) or accelerate up to the maximum one given by its geometry.\\
Following typical values for these physical parameters, for the simulation it has been chosen to set the maximum free-falling speed to 55 $\left[\si{\meter\per\second}\right]$ \cite{b8}, and a maximum falling one when the parachute is opened equal to 4.87 $\left[\si{\meter\per\second}\right]$, considering a mass of 100 $\left[\si{\kilogram}\right]$, a drag coefficient of 1.5 and a parachute's area of 45 $\left[\si{\meter^2}\right]$,  $\bar{v}_{z,max} = \sqrt{\frac{2 \, m \, g}{c_p \, \rho \, A}} = 4.87 \left[\si{\meter\per\second}\right]$ \cite{b9}. \\
Regarding the minimum falling velocity, a value equal to 25\% of the maximum one has been selected arbitrarily because it is hard to define without real experiments.\\
Even though a physical system able to provide the required displacement on the plane has not been identified, some minimal performances in movement and errors in the actuation have been considered.

\subsection{Unicycle-like parachute model} \label{unicyle}
A more sophisticated analysis can consider that a real parachute can brake the fall and steer by toggling the lines connecting the sail and the payload. At the same time, it has only a forward velocity. These considerations suggest that it can be modelled as a unicycle, with equations given by (\ref{eq:NL}).
\begin{equation}
\label{eq:NL}
    \begin{gathered}
       x_{i+1} = x_i + V\cos{\theta}\Delta t + \nu_x  \\
       y_{i+1} = y_i + V\sin{\theta}\Delta t  + \nu_y\\ 
       z_{i+1} = z_i + \bar{v_z}\Delta t + v_z\Delta t + \nu_z\\ 
       \theta_{i+1} = \theta_i + \omega\Delta t + \nu_{\theta} 
    \end{gathered}
\end{equation}
Hence, the control inputs are the forward and the rotational velocity and the falling breaks, named $V$, $\omega$, and $v_z$.
Even though in a real parachute, a minimum forward velocity is necessary to ensure the lift force sustaining the fly, in this project, the forward velocity has been assumed to be bounded between 0 and a maximum value to simplify the control laws.\\
For what concerns the boundaries of the inputs, a maximum forward velocity of 13 $\left[\si{\meter\per\second}\right]$ has been selected \cite{b7} \cite{b10}, while the maximum rotation speed is 0.52 $\left[\si{\radian\per\second}\right]$. As for the control in the vertical direction, the same holds from the linear model.

\subsection{Communication system}\label{communication}
The adopted communication system has to allow bidirectional data exchange between chutes moving in the space. For example, this can be achieved by using a UWB mounted on the payload, which has a typical communication range of 50 $\left[\si{\meter}\right]$ \cite{b11}.\\
The measurement of the absolute position is done using a GPS, with an uncertainty of 5 $\left[\si{\meter}\right]$ while for what concerns the relative measurement, multiple technologies can be used such as LIDAR, stereo camera or the combination of a UWB and a camera. For the seek of the thesis, it is assumed that the relative measurement can be done when the agents are closer than 50 $\left[\si{\meter}\right]$ and are provided with an uncertainty of 1 $\left[\si{\meter}\right]$. It is also assumed that each agent can measure its orientation with respect to a fixed direction but cannot measure the others'.